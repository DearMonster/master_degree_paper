%# -*- coding: utf-8-unix -*-
%%==================================================
%% conclusion.tex for SJTUThesis
%% Encoding: UTF-8
%%==================================================

\begin{summary}

这里是全文总结内容。

在前面的章节,我们已经介绍了面向分类应用的差分隐私算法在范围计数查询需求中存在的噪音线性叠加的问题,探讨了问题的频率矩阵加噪模型本质。接着详细介绍基础算法DiffGen与基于直方图发布方式的优化方案,DiffGen的匿名化过程决定其拥有一致性特性,这是二者之间的联系基础。然后说明DiffGen具有一维直方图特性,从树状组织结构和一致性约束出发,设计并实现优化算法DiffCon。在DiffCon中,设计了新的应答查询模式和全局敏感性定义,并调整噪音分布已达到在单位和范围查询情况均获得较好性能的目的。最后从理论角度,由误差方差对优化性能做了讨论总结。    

\end{summary}
