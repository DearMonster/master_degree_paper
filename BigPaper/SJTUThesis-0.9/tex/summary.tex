%# -*- coding: utf-8-unix -*-
%%==================================================
%% conclusion.tex for SJTUThesis
%% Encoding: UTF-8
%%==================================================

\chapter{总结与展望}

\section{创新点总结与核心工作介绍}

本课题研究的核心工作在于:
\begin{enumerate}
	\item 事物总有好坏两面性,高速发展的数据应用技术伴随着隐私隐患问题。在此背景下,本课题立足于具有最高隐私保障的差分隐私技术的解决思路,进一步探究隐私保护过程中的数据可用性损失问题。分析总结了隐私保护、决策树算法、匿名技术、数据维度问题的相关背景和研究现状,为后续章节的论述做好铺垫。
	\item 总结面向分类的应用场景,提出课题的核心目标---优化非交互式差分隐私匿名算法的发布数据精度,介绍差分隐私相关的背景知识与精度量化式,并明确难点问题的本质模型:频率矩阵加噪模型。
	\item 基于经典算法在范围计数查询需求中的性能瓶颈,详细描述了基础算法的匿名化过程和关键实现细节,并概括其缺陷特征。针对问题寻找解决方案,探讨一维直方图发布算法的一致性约束优化思想,总结初步的解决思路。
	\item 就基础算法存在的问题正式提出改进思路,详细论述课题算法DiffCon的设计和实现。首先总结相关特征命题,证明基于一维直方图特性使用一致性约束优化方案的正确性;接着阐述具体的实现:构造辅助树、重定义应答查询模式、实现噪音分布优化算法,并说明主要实现类图;然后,通过满足一致性约束的最小二乘法目标式展示优化思路的理论核心,并证明线性规划的最优化命题正确性;最后从理论分析角度探讨DiffCon算法性能。
	\item 借助真实数据集与分类器,针对课题的优化差分隐私算法发布数据集准确度的目标,从多个角度设计测试实验。对于范围计数查询中的单位长度查询、范围查询,以及面向分类应用真实场景的测试,设计DiffCon与基础算法、直方图发布方式的比对实验,同时分析并讨论实验现象。结果证实了本课题优化算法较基础算法在发布数据可用性、准确度扩展性上具有较高的提升,与理论分析结果相符,实现了非交互式差分隐私框架下发布数据的高可用性。
\end{enumerate}	

总结本课题研究的创新点包括:

\begin{enumerate}
	\item 总结频率矩阵加噪模型。经算法过程研究分析与范围查询的实验结果,总结了非交互式差分隐私框架基于频率矩阵处理数据方式中存在的问题本质模型,从噪音敏感性与应答查询模式两个角度,揭示了基础算法和直方图发布方式中性能瓶颈的根本原因所在。
	\item 面向分类应用的差分隐私算法设计。明确分类应用中范围计数查询的需求特性,通过基于决策树构建算法的数据处理过程,设计对此领域具有更好应用性能的差分隐私算法。
	\item 匿名化处理与一致性约束优化思路。匿名化是实现一致性特性的途径,通过匿名化技术,使用树结构组织属性泛化关系,从而使得发布数据集在辅助树上具有一致性约束关系。(1)这是基于TMSC算法设计的应答查询模式的基础,保证了应答节点集在噪音叠加总量上的最优化目标。(2)这是噪音分布调整算法的基础,结合最小二乘法思想满足线性规划的最优目标式。(3)这是通过仅发布叶节点数据项就能达到TMSC算法目的的基础,同时避免了最小顶点集的搜索过程,并且保护了辅助树信息,不失安全性。
	\item 提出满足一致性约束的最小二乘法目标式。噪音分布调整算法不仅仅是简单的噪音数值调整,它是实现优化理论的关键组成部分之一,具有最小二乘法思想的算法设计,拥有严谨的一致性约束理论支持。
	\item 采用真实数据集与分类器的实验设计。实验部分基于真实数据与场景,而不是仿真模拟。因此,它不仅有力地验证了理论性能分析的正确性,并且呈现了理论所无法推导的细节现象,将加深对课题算法性能的研究分析。
\end{enumerate}	
\section{后续研究工作}

本小节将在解决方案背景、算法特性、实现细节等多个方面详细分析本课题存在的不足,并在不足的基础上提出后续研究工作的规划。

\begin{enumerate}
	\item 课题算法是面向分类应用的范围计数查询需求,此针对性较强、适用性较窄,未探究分类应用中的其他类型查询需求,以及分类学习以外的其他数据挖掘算法模型。
	
	改进方案:在后续研究中可通过实验设计探究课题算法在其他类型查询需求中的性能指标,并且也可设计线性回归、概率学习等数据挖掘算法实验,探究其适用性。
	\item DiffCon算法运行过程中需要维护众多的匿名树结构与算法辅助树,在高维度数据集情况下,对存储空间复杂度和处理时间复杂度上产生较大的负担。更高的CPU、I/O或内存性能只能治本,无法从根本上解决这些辅助结构带来的开销,使得算法的绝大部分消耗花在辅助结构的处理上,核心的数据集处理仅占据一小部分,有点本末倒置。
	
	改进方案:在后续研究中应该设计更好的辅助处理手段,替代或者优化树状的数据结构基础,改进高维度数据集中的匿名树和辅助树的存储、处理复杂度。
	\item 关于范围查询的实验设计还应扩展。通过顺序读入叶节点来表示查询范围扩增的做法,对范围查询性能总结而言略显粗糙,因为真实场景中范围的扩增不仅仅包含顺序递增这一方式。
	
	改进方案:可设计随机读入不重复叶节点的方式,或者直接选取非叶结点作为范围测试的做法,丰富这部分的实验。
	\item 面向真实分类应用的实验设计应扩展。仅通过一种分类器的实验测试略显单薄,对于真应用场景应予以更多的测试。
	
	改进方案:追加真实应用场景的实验,除了分类应用外,还可设计线性回归、广告推荐等相关实验。
\end{enumerate}	