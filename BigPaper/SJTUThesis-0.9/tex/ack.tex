%# -*- coding: utf-8-unix -*-
\begin{thanks}

已然2016年了,光阴悄无声息,不留痕迹,此刻在学位论文的末页回首曾经,心中五味杂陈。三年前我是个毛毛躁躁的少年,向往诗和远方;现在的我正值弱冠之年,恪守初衷,不再年少。

这三年匆匆而过,可谓宠辱不惊。收获了学识和情谊,却也有丢失,好在工作和毕业皆在眼前。这几年有太多的问题我没想通,我需要时间好好思考,慢慢感悟。本科毕业季我选择了分手,直到现在才明白我曾经一直是别人的依靠;我也曾有过一部仅能存200条短信的手机,但我现在再也没找到那些能和我发满短信的人;印象里的父亲一直是个超人般的存在,十八般家电样样精通,可今年暑假却看到他眉角的点点老人斑。时间能告诉我们一切,但需要耐心。有幸我在交大得到了缓冲。

我出生于南方的一个山水小城,求学20余载,离家7年,回家次数寥寥可数,但说实话我从未觉得离家远。家是种牵挂,牵挂你的人和你牵挂的人,致我的伟大父母。从未曾多虑过我以后会是个怎样的父亲,也未曾评定在一个小县城当了一辈子公务员的老爸事业成就如何,但别人问我偶像是谁的时候,他当仁不让。25年来他只教了几样东西给我:乐观,大气,责任。我始终相信没有解决不了放不下的问题,也从未缺过交心的朋友,从小未曾感受过物质上低人一等,接受的教育一直是同时期最好的,深谙孝为先的理念。若我也能给我将来的小孩这些礼物,请大家无情地夸我。对比于父亲鼓励式的教育,很庆幸有母亲相得益彰的配合,让我懂得中国人低调谦虚的美德。从我上初中起,她便全程在家照顾我了,每天最大的工作就是每餐饭都有儿子的私房菜,这习惯延续至今。我母亲这20年间都在为一件事努力,让回家的儿子丈夫感受家的暖意,就像她所说的,我和老爸过好了她就很好。责任另一半的含义是她告诉我的。

回顾我的求学之路,多赖诸多老师的帮助。感谢小学时代的兰建斌老师,第一个男性班主任让我的性格开朗了太多;感谢初一的陈昇老师,在一心习文的时代支持我“文武发展”,延续篮球梦;感谢高中班主任施仁港老师,恬静致远仁义为师,在文学上指荐让我受益至今;感谢大学时代的彭舰老师的知遇之恩,让我在大四低迷之际重整旗鼓。

感谢我的导师梁阿磊老师,传道授业解惑,他给了我巨大的帮助。治学严谨,为人正直是他给我最深刻的性格印象。依稀记得来实验室的初次亮相,由于关键技术要点表达的含糊不清受到了他的批评。对事不对人,我想我遇到了个好老师。在学生培养上,他的观点可谓点醒梦中人:“我们不是培养工程师,而是培养聪明的人”。读书这么多年,方始思考读书的意义所在。本专业知识教会我们方法论的基本思路和规律,举一反三、走出书本,高等学府培养的不仅是专才,更应该是具有高水平处事能力的人。“敢于接受问题,不慌不忙地思考并解决”,这是我们实验室届届相传的思想精髓。对于这个年纪慌里慌张、浮华躁动的我来说,感触颇深。梁老师时常把他的人生领悟和我们做交流,比如健康。正值青春、冉冉升起的我们顾及不到20年后的生活状态,但梁老师通过跑马拉松和骑行活动,带领我们感受运动乐趣,对比与当下的“挣得多”,提前比拼“活得久”。

我即将要直面生活、肩负责任,跟随梁老师的3年时间里,我在世界观、人生观上获益满满。

感谢指导我学术研究的姚建国老师,踏实奋进、勤勉进取、对学生负责,交大的未来很需要这样的老师。我的小论文和毕设是在姚老师的指导下完成的,很惭愧的是伴随着督促,我的不努力有点辜负了他的关心。和他研二一年的密切相处中,他对学术的执着追求和高效的时间利用率给了我极大的触动,任何成果是和努力分不开的。衷心祝福他硕果累累,实验室的发展越来越好。感谢袁明轩博士对我科研工作的指导和关心,在学术问题上及时帮助我答疑解惑,使我收益颇丰。

感谢管海兵老师的大力支持,给我提供了如此优异的科研学习环境。感谢李建老师、马汝辉老师对我的指导和帮助。感谢戚正伟老师的帮助和照顾。

感谢我的师兄谢涛、张雕、曹亮、朱敏君、刘宇宸、周仪、成立、宋涛、周凡夫、林立伟、吴杰蔚、夏永峰在实验室生活和找工作上给了我许多帮助和关心,也让我认识到自己的不足,学习到了很多东西。感谢我可爱的师弟何书胜、王坤、唐西铭、陈天翔、刘义康、何西迪、付周望、孙炜程、庄奕锋和师妹胡力里、蒋瑶瑶、陈丽霞,和你们的相处很快乐,你们对我的支持很给力。感谢我的战友邓煜,你的鼓舞是我不竭的动力。愿实验室越办越好,友谊天长地久。

感谢我的朋友赵耀、何俊杰、张思明、王玮、吴嘉雯、张世晨、范嘉华、朱维、王佳俊、陈浩、丁筱原、黄斌、苏仰旭、林巧容、肖莉,还有许许多多朋友一时之间难以尽数,与你们的友谊使我从不感觉孤单,能与你们相识是我的幸运,愿友谊长存。

感谢所有帮助和关心过我的人。祝大家前程似锦,生活安康,万事如意!



\end{thanks}
