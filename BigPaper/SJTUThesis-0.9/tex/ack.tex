%# -*- coding: utf-8-unix -*-
\begin{thanks}

已然2016年了,在交大的三年匆匆而过,光阴从不留痕迹,唯有回首之时深感人生苦短。从不觉得岁月的痕迹会和我有过什么交集,也从没想过人老珠黄之时我将过着怎么的生活,因为我正直青春,冉冉升起。但就像镜子能够正衣冠,我在自己身上看不到岁月的捶打,但却发生在我最亲密最爱的人身上,我真希望能把这些痕迹分点给我。今年暑假深圳行,居然看到了老爸的眉角有了点点老人斑。我是年轻,我活力十足,但那一刻我从未如此地惧怕过衰老,担忧下次的碰面。从前我体会不到和父母旅行游玩的乐趣,我相信有种东西叫代沟;现在,和他们的旅行让我觉得心安,踏实。每个人都可以在另外一个地方找到这种感觉,这个地方叫家。求学20余载,离家7年,回家次数寥寥可数,但我从未觉得离家远。那个字是个牵挂,牵挂你的人和你牵挂的人,足矣。

读了这么多年书,我很欣慰我没有死读书

和我聊得最多的是计划着下次出游,时间地点恨不得立刻定好。我们虽然看不见自己的样貌和变化,但有种叫做镜子的东西。我的确没老,依旧活力十足,但从我最爱最亲密的人身上看到了我以前
依稀记得2013年爸妈送我来上海的情景,每年回去

\end{thanks}
