%# -*- coding: utf-8-unix -*-
%%==================================================
%% abstract.tex for SJTU Master Thesis
%%==================================================

\begin{abstract}

高速发展的数据应用技术让数据的潜在价值得到了充分的利用,但是新的数据挖掘模式和攻击手段的出现使得传统的隐私保护方式变得不那么安全可靠了。一方面,数据拥有者在发布数据时需要对隐私信息进行特殊处理;另一方面,发布后的数据可能会面对各类数据挖掘算法,此时也应该保障隐私安全。对数据拥有者而言,对于无法预计的攻击模式和数据挖掘算法,如何设计强力的隐私保护算法成了难点问题。

针对上述情况,差分隐私技术从根本上解决了这个问题。它不关心具体的应用背景,即使在最坏的情况下攻击者已经掌握了除了某条记录之外的所有记录信息,也无法推断出该条记录的隐私信息情况。但基于失真技术的加噪操作会使得差分隐私技术在保护隐私的同时影响了数据的可用性,降低了后续应用的分类准确度,这就对差分隐私算法的设计提出了新的要求。

本文针对差分隐私技术在数据挖掘分类算法中的应用,就范围计数查询引起的噪音叠加问题,在经典算法的基础上设计一种基于树状结构、利用一致性特性优化噪音分布的非交互式差分隐私优化算法DiffCon,使得在保障数据隐私的前提下,有效提升发布数据的可用性。

%本文的主要工作包括:1)分析现有的面向分类应用的差分隐私匿名算法,总结较高的查询维度会对产生的噪音线性叠加问题成为影响算法可用性的关键因素。2)研究经典算法与后续的数据挖掘分类应用间的联系,概括其缺陷模型。%查询维度相关的噪音叠加现象制约了算法扩展性的提升。
%3)基于重匿名属性值间的一致性特性,设计并实现DiffCon算法以优化以行项为单位的独立噪音分布和应答的处理方式。
%4)经真实数据集测试算法性能并分析实验结果---DiffCon提升了发布数据的分类准确度并具有较好的扩展性。

本文的主要工作包括:
1)就隐私保护、匿名算法和数据挖掘分类应用领域间存在的问题和联系,引出课题主题——面向分类应用的差分隐私技术,旨在兼顾强力隐私保护力度和发布数据的高可用性。
2)研究并明确本课题所要解决的问题——在传统的基于频率矩阵模型的差分隐私算法中,以行项为单位的独立加噪和应答处理模式导致噪音的线性叠加,制约了发布数据在较高查询维度请求中的可用性。
3)针对问题提出优化思路,基于经典算法的一维直方图特性和一致性约束关系,设计并实现非交互式差分隐私匿名算法DiffCon。
4)经真实数据集测试算法性能并分析实验结果---对比与传统方案,DiffCon在发布数据的分类准确度和扩展性上提升效果显著。


%现在各种数据面临的问题:1,发布的数据要不能泄露隐私。2,经得起各种数据挖掘分析。
%关注点:1,如何满足差分隐私。2,发布出的数据可用性的保障。
%厉害之处:保 护 方 法 可 以 确保 在 某 一 数 据 集 中 插 入 或 者 删 除 一 条 记 录 的 操 作不 会 影 响 任 何 计 算 的 输 出 结 果 另 外该 保 护 模 型不关心攻击者所具有的背景知识即 使 攻 击 者 已经掌握除某一条记录之外的所有记录的信息 该记 录 的 隐 私 也 无 法 被 披 露 
%两种数据保护框架:交互式和非交互式
%关键字:决策树,k匿名
%1,决策树的基本用法,fireman,然后匿名化得diffGen。 维度问题:主要指查询维度,从而有噪音叠加问题。 指出构造树和后续的决策树计数应用之间的联系。提出一致性


%    仅支持一维查询情况啊!!!  典型的就是计数查询  原始模型优化主题

\keywords{\large 隐私保护 \quad 差分隐私 \quad 决策树 \quad 匿名化 \quad 一致性 }
\end{abstract}

\begin{englishabstract}

fdfdfsdfsfdsfsdfsdfdsfsdfsdf

\englishkeywords{\large SJTU, master thesis, XeTeX/LaTeX template}
\end{englishabstract}

