%# -*- coding: utf-8-unix -*-
%%==================================================
%% abstract.tex for SJTU Master Thesis
%%==================================================

\begin{abstract}

高速发展的数据应用技术让数据的潜在价值得到了充分的利用,但是新的数据挖掘模式和攻击手段的出现使得传统的隐私保护方式变得不那么安全可靠了。一方面,数据拥有者在发布数据时需要对隐私信息进行保护处理;另一方面,发布后的数据会面对各类数据挖掘应用和隐私攻击的威胁。此时,如何设计具有强力隐私保障的算法成了难点问题。

针对上述情况,差分隐私技术从根本上解决了这个问题。它不关心具体的应用背景,即使在最坏的情况下攻击者已经掌握了除了某条记录之外的所有记录信息,也无法推断出该条记录的隐私信息情况。但它基于失真技术的加噪操作会影响数据的可用性,降低后续应用的分类准确度,这就对差分隐私算法的设计提出了新的要求。

本文针对差分隐私技术在分类应用中的范围计数查询需求,就查询维度引起的噪音叠加问题,在经典算法的基础上从一致性约束出发,基于最小二乘法目标式设计并实现非交互式差分隐私匿名算法DiffCon,使得在保障数据隐私的前提下,有效提升发布数据的可用性。

%本文的主要工作包括:1)分析现有的面向分类应用的差分隐私匿名算法,总结较高的查询维度会对产生的噪音线性叠加问题成为影响算法可用性的关键因素。2)研究经典算法与后续的数据挖掘分类应用间的联系,概括其缺陷模型。%查询维度相关的噪音叠加现象制约了算法扩展性的提升。
%3)基于重匿名属性值间的一致性特性,设计并实现DiffCon算法以优化以行项为单位的独立噪音分布和应答的处理方式。
%4)经真实数据集测试算法性能并分析实验结果---DiffCon提升了发布数据的分类准确度并具有较好的扩展性。

本文的主要工作包括:
1)背景介绍。就隐私保护、匿名算法和数据挖掘分类应用间的问题和联系,介绍课题背景和研究现状。%问题模型,---在面向分类应用的差分隐私技术,旨在兼顾强力隐私保护力度和发布数据的高可用性。
2)明确问题。在传统的基于频率矩阵模型的差分隐私算法中,独立的加噪方式和粗糙的查询应答模式导致了噪音的线性等额叠加问题,降低了发布数据在较高查询维度请求中的可用性。
3)描述解决方案。由经典算法基础和一致性约束优化方案出发,设计新的应答查询模式并重定义全局敏感性,立足于最小二乘法的理论目标式,设计并实现DiffCon算法。
4)实验测试。通过真实的数据集和分类器验证算法性能并进行实践测试,其实验结果表明DiffCon算法在发布数据的分类准确度上具有显著的提升效果。


\keywords{\large 隐私保护 \quad 差分隐私 \quad 一致性约束 \quad 决策树 \quad 匿名化 }
\end{abstract}

\begin{englishabstract}

The rapidly developing data applied technology takes full advantage of the potential value of data, but it's not so safe and reliable for traditional privacy preserving technology after the emerging technology of data mining and attack means on privacy. The data owner needs to protect private information while publishing datasets on the one hand, on the other hand, the released data is under threats of various data mining applications and attacks on privacy. At this moment, it is a difficult problem on how to design a algorithm with strong guarantee to privacy.

For the above conditions, differential privacy solved the problem fundamentally. It doesn't care about the specific applied background for released data. Even in the worst case that a attacker has acquired all of the records except for one, and the attacker couldn't infer the private information of that one. However, the distortion-based operation of adding noise in differential privacy affects the availability of released data, which reduces classification accuracy in the subsequent application. This is making new demands on the design of differentially private algorithm.

Aimed at range-count queries towards to classification applications in differential privacy, we design and realize a non-interactive differentially private anonymity algorithm DiffCon, which is based on a fundamental algorithm and a least square method with consistent constraint. DiffCon effectively enhance  the availability of released data, on the premise of privacy protection.

The main work of this dissertation includes:
1) Background introduction. Based on the relationship among privacy protection, anonymity algorithm and classification application in data mining, we introduce the project background and research status.
2) Problem statement. In the traditional differentially private algorithm towards to frequency matrix, the independent noise-adding way and the rough inquiry-response pattern cause linear superposition of noise by an equal amount. This problem reduce the usability of released data in higher query  dimension.
3) Solution description. Based on fundamental algorithm and least square method with consistent constraint, we design a new inquiry-response pattern and redefine sensibility to realize DiffCon algorithm.
4) Experimental evaluation. We validate the algorithm performance and then testing the classification accuracy by a real dataset and a classifier. The experiment results show that DiffCon significantly improves the classification accuracy and the query-dimension expansibility for released data.


\englishkeywords{\large Privacy Protection, Differential Privacy, Consistent Constraint, Decision Tree,  Anonymization}
\end{englishabstract}

