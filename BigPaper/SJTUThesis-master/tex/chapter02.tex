%# -*- coding: utf-8-unix -*-
%%==================================================
%% chapter02.tex for SJTU Master Thesis
%% Encoding: UTF-8
%%==================================================

%1,展开差分隐私定义,laplace,讨论维度问题
%2,展开面向决策树的差分隐私设计,典型方案


\chapter{相关背景技术}
\label{chap:background}

\section{引言}

上一章描述了课题所涉及的背景知识及相关问题的研究现状,本章将介绍理论知识及问题模型。主要内容为差分隐私相关的数学定义和重要概念介绍,以及频率矩阵加噪模型描述。

\section{差分隐私}

\subsection{差分隐私定义}

%给定数据集$D$,另一数据集$D'$和它至多相差一条记录。
在差分隐私中,我们希望某个在数据集上的运算结果不因任意一条记录的改变而变化明显。

\begin{defn}
	
($\varepsilon$\textsc{-差分隐私})\cite{Dwork Calibrating} 数据集$D$和$D'$至多相差一条记录,即$|D$$\triangle$$D'|$ $\leqslant$ 1。一个随机算法$\partial$满足$\varepsilon$-差分隐私,当且仅当对于$\partial$任意可能的输出$O$,我们有下式成立:

\begin{equation}

Pr[\partial(D) = O] \leqslant e^{\varepsilon} \cdot Pr[\partial(D') = O]

\end{equation}
其中,$Pr$表示事件发生的概率,在此也表示隐私被披露的风险。$\varepsilon$为隐私预算,$\varepsilon$越小,算法的隐私保护程度越高。

\end{defn}

实现差分隐私的主要方法是对真实值添加噪音扰动,而噪音量的量级大小是由全局敏感性(Sensitivity)来定义。

\begin{defn}
	(\textsc{全局敏感性}\cite{Dwork Calibrating}) 数据集$D$和$D'$至多相差一条记录,即$|D$$\triangle$$D'|$ $\leqslant$ 1。$\mathbb{F}$的查询维度为d,并且有$\mathbb{F}$:$D \rightarrow \mathrm{R}^d$。那么,函数$\mathbb{F}$的全局敏感性$S(\mathbb{F})$定义为:
\begin{equation}

	S(\mathbb{F}) = \max \limits_{D,D'} \| \mathbb{F}(D) - \mathbb{F}(D') \|_{1}
\end{equation}
其中,$\|\cdot\|$表示$L_{1}$范数。
\end{defn}

本课题涉及拉普拉斯噪音机制(Laplace mechanism)和指数机制(Exponential mechanism),接下来做简要介绍。

\subsection{差分隐私实现机制}

\section{频率矩阵}

\subsection{数据域尺寸}

\subsection{频率矩阵模型}
