%%==================================================
%% chapter05.tex for SJTU Master Thesis
%% based on CASthesis
%% modified by wei.jianwen@gmail.com
%% version: 0.3a
%% Encoding: UTF-8
%% last update: Dec 5th, 2010
%%==================================================

%\bibliographystyle{sjtu2} %[此处用于每章都生产参考文献]
\raggedbottom
\chapter{实验与分析}
\label{chap:evaluation}
%向下 或 向上 取整的小优化

% 1,单位长度的误差方差比对,3种  diffGen,已调整/未调整)。 设置高度,画5个图
% 横坐标 第i个叶子,纵坐标 标准差  x 4(4个高度,4,7,10,13)  1,1  0.5   递增,因为层数越高 代价越小 噪音越大
% 横坐标  高度,纵坐标平均每个叶子的标准差1,2

% 应该是 每个error(i),有new - baseline  与  noise - baseline。然后  sum[error(i)]

% 2,内部节点。一样 也是5个图
% 横坐标 第i个节点,纵坐标 标准差  x 4(4个高度,4,7,10,13)2,1
% 横坐标  高度,纵坐标平均每个节点的标准差2,2

% 3, 范围查询 。首先还是单点,内部节点的单点。比对 已调整/未调整 的差别,此时不比叶节点了
% 然后是范围查询情况,扯扯淡 因为没法设计啊  好像能设计诶 。 通过 c45肯定可以说明
% 横坐标  范围从2,3,4.。。。。。。,纵坐标,对应的标准差   5个图   3


% 4,找个推荐算法

\section{引言}

在前面的章节,我们已经介绍了面向分类应用的差分隐私算法在范围计数查询需求中存在的噪音线性叠加的问题,探讨了问题的频率矩阵加噪模型本质。接着详细介绍基础算法DiffGen与基于直方图发布方式的优化方案,DiffGen的匿名化过程决定其拥有一致性特性,这是二者之间的联系基础。然后说明DiffGen具有一维直方图特性,从树状组织结构和一致性约束出发,设计并实现优化算法DiffCon。在DiffCon中,设计了新的应答查询模式和全局敏感性定义,并调整噪音分布已达到在单位和范围查询情况均获得较好性能的目的。最后从理论角度,由误差方差对优化性能做了讨论总结。

本章节将在多个方面对DiffCon算法做比对实验,通过真实数据集测试算法准确度性能,以及基于$TMSC$算查询应答模式的噪音误差方差表现。主要工作包括:实验数据集、平台介绍;实验方案的设计;就实验结果进行分析说明。

\section{实验平台及设置}

\subsection{实验平台介绍}

\subsection{数据集简介}

\section{基于误差的测试与分析}

\subsection{单位长度查询} 

\subsection{范围查询} 

\section{面向分类算法的测试与分析}

\subsection{实验方案}

\subsection{实验结果及分析}    

\section{本章小结}

