\begin{thebibliography}{1}

\bibitem{Defining Privacy for Data}Clifton,Kantarcioglu M,Vaidya J.Defining privacy for data mining. Proceedingsofthe National Science Foundation Workshop on Next Generation Data Mining. USA,2002.

\bibitem{l-diversity}Machanavajjhala A, Kifer D, Gehrke J, et al. l-diversity: Privacy beyond k-anonymity. ACM Transactions on Knowledge Discovery from Data(TKDD), 2007.

\bibitem{面向数据库应用的隐私保护研究综述}周水庚, 李丰, 陶宇飞, 肖小奎. 面向数据库应用的隐私保护研究综述. 计算机学报, 2009, 32(5): 847-861. 

\bibitem{Dwork Calibrating}C. Dwork, F. McSherry, K. Nissim, and A. Smith, Calibrating noise to sensitivity in private data analysis, in Proceedings of {\it TCC }, 2006.

\bibitem{m-Invariance}Xiao X, Tao Y. M-invariance: towards privacy preserving re-publication of dynamic datasets. Proceedings of the 2007 ACM SIGMOD international conference on Management of data. ACM, 2007: 689-700.

\bibitem{clustering}Vaidya J, Clifton C. Privacy-preserving k-means clustering over vertically partitioned data. Proceedings of the ninth ACM SIGKDD international conference on Knowledge discovery and data mining. ACM, 2003: 206-215.

\bibitem{multidimensional k anonymity}LeFevre K, DeWitt D J, Ramakrishnan R. Mondrian multidimensional k-anonymity. Data Engineering, 2006. ICDE'06. Proceedings of the 22nd International Conference on. IEEE, 2006: 25-25.

\bibitem{Distributed Privacy}Rottondi C, Verticale G, Krauss C. Distributed privacy-preserving aggregation of metering data in smart grids. Selected Areas in Communications, IEEE Journal on, 2013, 31(7): 1342-1354.

\bibitem{k-anonymity} Sweney L. $k$-anonymity: A model for protecting privacy. International Journal on Uncertainty. Fuzines and Knowledge Based Systems. 1998.

\bibitem{decision tree}J. R. Quinlan. Induction of decision trees. Machine Learning, 1(1):81-106, 1986.

\bibitem{C45}J. R. Quinlan. C4.5: Programs for Machine Learning. Morgan Kaufmann, 1993.

\bibitem{cart}Breiman L, Friedman J, Stone C J, et al. Classification and regression trees[M]. CRC press, 1984.

\bibitem{compounding attack} Ganta S R, Kasiviswanathan S P, Smith A. Composition attacks and auxiliary information in data privacy. {\it Proceedings of the ACM SIGKDD}

\bibitem{background attack} Wong R C W,Fu A,Wang K,et al. Can the utility of anonymized data be used for privacy breaches. {\it ACM Transactions on Knowledge Discovery from Data },2011,5(3):16

\bibitem{Bottom-up generalization}Wang K, Yu P S, Chakraborty S. Bottom-up generalization: A data mining solution to privacy protection[C]//Data Mining, 2004. ICDM'04. Fourth IEEE International Conference on. IEEE, 2004: 249-256.

\bibitem{Top-down specialization}Fung B, Wang K, Yu P S. Top-down specialization for information and privacy preservation[C]//Data Engineering, 2005. ICDE 2005. Proceedings. 21st International Conference on. IEEE, 2005: 205-216.

\bibitem{SuLQ}Blum A, Dwork C, McSherry F, et al. Practical privacy: the SuLQ framework. Proceedings of the twenty-fourth ACM SIGMOD-SIGACT-SIGART symposium on Principles of database systems. ACM, 2005: 128-138.

\bibitem{diffp-c4.5}Friedman A, Schuster A. Data mining with differential privacy[C]//Proceedings of the 16th ACM SIGKDD international conference on Knowledge discovery and data mining. ACM, 2010: 493-502.

\bibitem{exponential}F. McSherry and K. Talwar. Mechanism design via differential privacy. In FOCS.

\bibitem{DiffGen}N. Mohammed, R. Chen, B. C. M. Fung, and P. S. Yu. Differentially private data release for data mining. In {\it SIGKDD }, 2011.

\bibitem{marginals}B. Barak, K. Chaudhuri, C. Dwork, S. Kale, F. McSherry, and K. Talwar, Privacy, accuracy and consistency too: A holistic solution to contingency table release, inPODS, 2007.

\bibitem{privbayes}J. Zhang, G. Cormode, C. M. Procopiuc, D. Srivastava, and X. Xiao. Privbayes: Private data
release via bayesian networks. InSIGMOD, pages 1423šC1434, 2014.

\bibitem{wavelet} X. Xiao, G. Wang, and J. Gehrke. Differential privacy via wavelet transforms. In {\it ICDE}, 2010.

\bibitem{boosting} M.Hay, V.Rastogi, G.Miklau, D.Suciu, Boosting the accuracy of differentially-private queries through consistency. In Proceedings of {\it VLDB},  2010.

\bibitem{dpcombination}McSherry F D. Privacy integrated queries: an extensible platform for privacy-preserving data analysis. Proceedings of the 2009 ACM SIGMOD International Conference on Management of data. ACM, 2009: 19-30.

\bibitem{adult}A. Frank and A. Asuncion. UCI machine learning repository, 2010.




\bibitem{meng2} Machanavajhala A, Gehrke J, Kifer D, Venkitasubramaniam M. $l$-diversity: Privacy beyond $k$-anonymity. In ICDE.

%\bibitem{meng3} Li N,Li T. $t$-closenes: Privacy beyond $k$-anonymity and $l$-diversity. Procedings of the 23rd International Conference on Data Enginering(ICDE).Istanbul,Turkey.

\bibitem{meng4} Wong R C W, Li J, Fu A W, Wang K.($\alpha$,$k$)-anonymity: An enhanced $k$-anonymity model for privacy-preserving data publishing. In SIGKD.





\bibitem{Dwork2}B. Barak, K. Chaudhuri, C. Dwork, S. Kale, F. McSherry, and K. Talwar. Privacy, accuracy, and consistency too: a holistic solution to contingency table release. PODS, 2007.





\bibitem{sparse_data_summary}Cormode G., Procopiuc C., Srivastava D., Tran T. T. Differentially private summaries for sparse data. In {\it ICDT}. ACM, 2012.

%\bibitem{sparse data} A. Narayanan and V. Shmatikov. Robust de-anonymization of large sparse datasets. In {\it IEEE Symposium on Security and Privacy}, pages 111–125, 2008.









\bibitem{max} L. Breiman, J. H. Friedman, R. A. Olshen, and C. J. Stone. Classification and Regression Trees. Wadsworth, 1984.

\bibitem{sparse_data}G. Cormode, M. Procopiuc, D. Srivastava, and T. Tran, "Differentially
private publication of sparse data", in Proceedings of {\it ICDT}, 2012.

\end{thebibliography}